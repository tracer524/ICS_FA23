\documentclass[UTF8]{ctexart}
\ctexset{
  section={
    format=\raggedright\zihao{3},
    name={T,}
  },
  subsection={
    format={\zihao{4}},
    number={\arabic{subsection}}
  }
}
\usepackage[a4paper,left=3cm,right=3cm,top=2cm]{geometry}
\usepackage{amsmath}
\usepackage{enumitem}
\usepackage{float}
\usepackage{threeparttable}
\usepackage{caption}
\usepackage{multirow}
\usepackage{graphicx}
\usepackage{listings}
\usepackage{color}

\definecolor{dkgreen}{rgb}{0,0.6,0}
\definecolor{gray}{rgb}{0.5,0.5,0.5}
\definecolor{mauve}{rgb}{0.58,0,0.82}
\lstset{frame=tb,
  language=C,
  aboveskip=3mm,
  belowskip=3mm,
  showstringspaces=false,
  columns=flexible,
  basicstyle={\small\ttfamily},
  numbers=left,%设置行号位置none不显示行号
  %numberstyle=\tiny\courier, %设置行号大小
  numberstyle=\color{gray},
  keywordstyle=\color{blue},
  commentstyle=\color{dkgreen},
  stringstyle=\color{mauve},
  breaklines=true,
  breakatwhitespace=true,
  escapeinside=`,%逃逸字符(1左面的键),用于显示中文例如在代码中`中文...`
  tabsize=4,
  extendedchars=false %解决代码跨页时,章节标题,页眉等汉字不显示的问题
}

\setlength\lineskiplimit{5.25bp}
\setlength\lineskip{5.25bp}

\title{ICS Homework 4}
\author{崔士强 PB22151743}
\date{November 10, 2023}

\bibliographystyle{plain}

\begin{document}

\maketitle
\section{}  % T1
\begin{enumerate}
  \item 0101 011 010 1 00100
  \item 0101 011 010 1 01100
  \item 0101 011 010 1 11111
  \item We can not aquire 0100 0000 by sign-extending a 5-bit number.
\end{enumerate}
\section{} % T2
\section{} % T3
\begin{enumerate}
  \item LDR R2, R1, \#0
  
        STR R2, R0, \#0
  \item MAR <- SR
  
        MDR <- Memory[MAR]
        
        MAR <- DR

        Memory[MAR] <- MDR
\end{enumerate}
\section{} % T4
0101 011 001 000 011

0101 100 010 000 100
\section{} % T5
Addressing modes: immediate, register, PC-relative, indirect, Base+offset.
\begin{table}[H]
  \centering
  \begin{tabular}{ccc}
    \hline\hline
    Category & Instructions & Addressing modes \\
    \hline
    \multirow{3}{*}{Operate} & ADD & register, immediate \\
                             & NOT & register \\
                             & LEA & register, PC-relative \\
    \hline
    Data movement & LDR & register, Base+offset \\
    \hline
    Control & JMP & register \\
    \hline\hline
  \end{tabular}
\end{table}
\section{} % T6
\begin{enumerate}
  \item AND 
  \item AND R3, R3, \#0
  \item NOT R1, R7

        ADD R1, R1, \#1

        ADD R1, R1, R6
  \item 
  \item ADD R1, R1, \#0
\end{enumerate} 

\section{} % T7
JMP: 2 accesses (One for the instruction and another for the address stored in BaseR)

ADD: 3 accesses (One for the instruction and two for the operands)

LDI: 3 accesses (One access to fetch the LDI instruction itself.
Another access to fetch the address from memory.
A third access to fetch the actual data from the location specified by the address.)
\section{} % T8
\begin{enumerate}
  \item x70A4
  \item Suppose the instruction is at memory address x3010
  
        1110 1100 0011 1101
\end{enumerate}
\section{} % T9
In all 16 bits of the value in R5, there are 4 bits which are 1.
\end{document}
\iffalse
\begin{figure}[h]
    \centering
    \includegraphics[scale=0.5]{name.png}
    \caption{name}
\end{figure}
\fi
