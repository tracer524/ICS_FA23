\documentclass[UTF8]{ctexart}
\usepackage[a4paper,left=3cm,right=3cm,top=2cm]{geometry}
\usepackage{amsmath}
\usepackage{enumitem}
\usepackage{float}
\usepackage{threeparttable}
\usepackage{caption}
\usepackage{multirow}
\usepackage{graphicx}
\usepackage{listings}
\usepackage{xcolor}
\usepackage{amssymb}
\ctexset{
  section={
    format=\raggedright\zihao{3},
    name={A,}
  },
  subsection={
    format={\zihao{4}},
    number={\arabic{subsection}}
  }
}
\renewcommand{\figurename}{Figure}
\definecolor{dkgreen}{rgb}{0,0.6,0}
\definecolor{gray}{rgb}{0.5,0.5,0.5}
\definecolor{mauve}{rgb}{0.58,0,0.82}
\lstdefinelanguage{LC3}{
  morekeywords={ADD, AND, BR, BRn, BRnz, BRnzp, BRnp, BRz, BRzp, BRp, JMP, JSR, LD, LDI, LDR, LEA, NOT, ST, STI, STR, PUSH, FILL, BLKW, STRINGZ, TRAP, RET},
  sensitive=false,
  morecomment=[l]{;},
  morestring=[b]",
  morekeywords=[2]{R0, R1, R2, R3, R4, R5, R6, R7}, % Define register keywords
}
\lstset{frame=tb,
  language=LC3,
  aboveskip=3mm,
  belowskip=3mm,
  showstringspaces=false,
  columns=flexible,
  basicstyle={\small\ttfamily},
  numbers=left,%设置行号位置none不显示行号
  numberstyle=\tiny\courier, %设置行号大小
  numberstyle=\tiny\color{gray},
  keywordstyle=\color{mauve},
  keywordstyle=[2]\color{purple},
  commentstyle=\color{dkgreen},
  stringstyle=\color{mauve},
  breaklines=true,
  breakatwhitespace=true,
  escapeinside=`,%逃逸字符(1左面的键),用于显示中文例如在代码中`中文...`
  tabsize=4,
  extendedchars=false %解决代码跨页时,章节标题,页眉等汉字不显示的问题
}

\setlength\lineskiplimit{5.25bp}
\setlength\lineskip{5.25bp}

\title{ICS Homework 6}
\author{崔士强 PB22151743}
\date{December 23, 2023}

\bibliographystyle{plain}

\begin{document}

\maketitle

\section{}  %T1
\begin{enumerate}
  \item 65 times.
  \item $x0BFB + \sum_{i=x0001}^{x0041} i$
\end{enumerate}

\section{}  %T2
\begin{enumerate}
  \item Using RET would result in the program returning to whatever address is in R7, which might not be the correct return address from the interrupt.
  \item If RET is used instead of RTI, PSR is not correctly restored.
\end{enumerate}

\section{}  %T3
\begin{enumerate}
  \item The character whose ASCII value is x0049(i.e. I).
  \item No. Value in R7 is lost after the instruction \lstinline{JSR B}, which means subroutine A cannot return as expected.
\end{enumerate}

\section{}  %T4
X is ACV(Access Control Violation) exception.

X is set to 1 when the value stored in MAR is greater than xFE00 or less than x3000 and PSR[15] $= 1$.

\section{}  %T5
\begin{enumerate}
  \item Read KBSR. If the first bit of KBSR is 0(i.e. value in KBSR is not negative), there is a new character input from the keyboard.
  \item The character is stored in KBDR. So read the KBDR.
  \item Keep checking the DSR until the value is not negative, then store the character in DDR.
  \item \ 
  
\begin{lstlisting}
START   LDI     R1, KBSR    ; Test for character input
        BRzp    START
        LDI     R0, KBDR
ECHO    LDI     R1, DSR     ; Test for display
        BRzp    ECHO
        STI     R0, DDR
        BRnzp   CONTINUE
KBSR    .FILL   xFE00       ; Address of KBSR
KBDR    .FILL   xFE02       ; Address of KBDR
DSR     .FILL   xFE04       ; Address of DSR
DDR     .FILL   xFE06       ; Address of DDR
\end{lstlisting}
\end{enumerate}

\section{}  %T6
\begin{enumerate}
  \item
\begin{lstlisting}
  BRzp    WAIT
  BRzp    LETTER
  BR      CONTINUE
  BRp     GETCHAR
  C1      .FILL     #-65
  C2      .FILL     #17
\end{lstlisting}
  \item Left shift \lstinline{R0} by 4 bits.
  \item No. The content of \lstinline{R0} has already been wiped out.
\end{enumerate}
\section{}  %T7
\begin{enumerate}
  \item \lstinline{H3ll0_W0r1d}
  \item 34.
\end{enumerate}

\section{}  %T8
\begin{enumerate}
  \item A character may be read more than once.
  \item 
  \item The first one. It's almost certain that the user cannot strike the keyboard in sync with the processor.
\end{enumerate}

\section{}  %T9
\lstinline{F !}

\section{}  %T10

\end{document}
\iffalse
\begin{figure}[H]
    \centering
    \includegraphics[scale=0.5]{name.png}
    \caption{name}
\end{figure}
\fi
