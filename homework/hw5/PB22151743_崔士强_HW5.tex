\documentclass[UTF8]{ctexart}
\usepackage[a4paper,left=3cm,right=3cm,top=2cm]{geometry}
\usepackage{amsmath}
\usepackage{enumitem}
\usepackage{float}
\usepackage{threeparttable}
\usepackage{caption}
\usepackage{multirow}
\usepackage{graphicx}
\usepackage{listings}
\usepackage{xcolor}
\usepackage{amssymb}
\ctexset{
  section={
    format=\raggedright\zihao{3},
    name={A,}
  },
  subsection={
    format={\zihao{4}},
    number={\arabic{subsection}}
  }
}
\renewcommand{\figurename}{Figure}
\definecolor{dkgreen}{rgb}{0,0.6,0}
\definecolor{gray}{rgb}{0.5,0.5,0.5}
\definecolor{mauve}{rgb}{0.58,0,0.82}
\lstdefinelanguage{LC3}{
  morekeywords={ADD, AND, BR, BRn, BRnz, BRnzp, BRnp, BRz, BRzp, BRp, JMP, JSR, LD, LDI, LDR, LEA, NOT, ST, STI, STR, PUSH, FILL, BLKW, STRINGZ, TRAP, RET},
  sensitive=false,
  morecomment=[l]{;},
  morestring=[b]",
  morekeywords=[2]{R0, R1, R2, R3, R4, R5, R6, R7}, % Define register keywords
}
\lstset{frame=tb,
  language=LC3,
  aboveskip=3mm,
  belowskip=3mm,
  showstringspaces=false,
  columns=flexible,
  basicstyle={\small\ttfamily},
  numbers=left,%设置行号位置none不显示行号
  numberstyle=\tiny\courier, %设置行号大小
  numberstyle=\tiny\color{gray},
  keywordstyle=\color{mauve},
  keywordstyle=[2]\color{purple},
  commentstyle=\color{dkgreen},
  stringstyle=\color{mauve},
  breaklines=true,
  breakatwhitespace=true,
  escapeinside=`,%逃逸字符(1左面的键),用于显示中文例如在代码中`中文...`
  tabsize=4,
  extendedchars=false %解决代码跨页时,章节标题,页眉等汉字不显示的问题
}

\setlength\lineskiplimit{5.25bp}
\setlength\lineskip{5.25bp}

\title{ICS Homework 5}
\author{崔士强 PB22151743}
\date{December 3, 2023}

\bibliographystyle{plain}

\begin{document}

\maketitle

\section{}  %T1
\lstinline{.END} pseudo-op is a sign to indicate that assembly process 
ends here. It is not an instruction. \lstinline{HALT} is an instruction
which is actually performed.

\section{}  %T2
A data structure following the FIFO rule, which means that if an element A
gets in the queue before B, A also gets out of the queue before B.

\section{}  %T3
\lstinline{LD} and \lstinline{ST} instructions expect memory locations, 
not immediate values as the second operand.

Fix: use \lstinline{LDI} and \lstinline{STI} instead.

\section{}  %T4
There is a problem: the linker won't recognize that the \lstinline{AGAIN} label in one module is intended to refer to the same entity as the \lstinline{AGAIN} label in the other module.
\section{}  %T5
\begin{enumerate}
  \item x0FFF
  \item No. Data in R2 will be saved in memory location x3004.
  \item Data in R2(i.e. x0FFF) will be performed as an operation, which is \lstinline{BRnzp #-1}. 
  Therefore the program will repeatedly perform this operation.
\end{enumerate}

\section{}  %T6
\lstinline{.FILL} initializes a memory location as its operand. 

\lstinline{.BLKW} sets aside $N$ memory locations, where $N$ is the operand.

\lstinline{.STRINGZ} initializes $N+1$ memory locations with a string and NULL, where $N$ is the length of the string.
\section{}  %T7
a: \lstinline{NOT R2, R0}

b: \lstinline{ADD R2, R2, #1}

c: \lstinline{BRz DONE}

d: \lstinline{ADD R0, R0, #1}
\section{}  %T8

\section{}  %T9
\begin{enumerate}
  \item AH
  \item After \lstinline{PUSH F} or \lstinline{PUSH G}
  \item Nothing
\end{enumerate}

\section{}  %T10
\begin{lstlisting}
PEEK:
  LD    R1, TOP_OF_STACK      ; Load the top of the stack address into R1
  ADD   R2, R1, #0            ; Copy the address to R2 for underflow checking
  
  BRzp  NO_UNDERFLOW          ; Branch if the stack is not empty
  
  LEA   R0, UNDERFLOW_MSG     ; Load the address of the underflow error message
  TRAP  x22                   ; Output the error message
  TRAP  x25                   ; Halt
  
NO_UNDERFLOW:
  LDR   R0, R1, #0            ; Load the value at the top of the stack into R0
  RET

TOP_OF_STACK:     .FILL x3FFF
UNDERFLOW_MSG:    .STRINGZ "Stack underflow error"
\end{lstlisting}

\end{document}
\iffalse
\begin{figure}[H]
    \centering
    \includegraphics[scale=0.5]{name.png}
    \caption{name}
\end{figure}
\fi
