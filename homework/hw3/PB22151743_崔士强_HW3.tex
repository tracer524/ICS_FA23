\documentclass[UTF8]{ctexart}
\ctexset{
  section={
    format=\raggedright\zihao{3},
    name={T,}
  },
  subsection={
    format={\zihao{4}},
    number={\arabic{subsection}}
  }
}
\usepackage[a4paper,left=3cm,right=3cm,top=2cm]{geometry}
\usepackage{amsmath}
\usepackage{enumitem}
\usepackage{float}
\usepackage{threeparttable}
\usepackage{caption}
\usepackage{multirow}
\usepackage{graphicx}
\usepackage{listings}
\usepackage{color}

\definecolor{dkgreen}{rgb}{0,0.6,0}
\definecolor{gray}{rgb}{0.5,0.5,0.5}
\definecolor{mauve}{rgb}{0.58,0,0.82}
\lstset{frame=tb,
  language=C,
  aboveskip=3mm,
  belowskip=3mm,
  showstringspaces=false,
  columns=flexible,
  basicstyle={\small\ttfamily},
  numbers=left,%设置行号位置none不显示行号
  %numberstyle=\tiny\courier, %设置行号大小
  numberstyle=\color{gray},
  keywordstyle=\color{blue},
  commentstyle=\color{dkgreen},
  stringstyle=\color{mauve},
  breaklines=true,
  breakatwhitespace=true,
  escapeinside=`,%逃逸字符(1左面的键),用于显示中文例如在代码中`中文...`
  tabsize=4,
  extendedchars=false %解决代码跨页时,章节标题,页眉等汉字不显示的问题
}

\setlength\lineskiplimit{5.25bp}
\setlength\lineskip{5.25bp}

\title{ICS Homework 3}
\author{崔士强 PB22151743}
\date{\today}

\bibliographystyle{plain}

\begin{document}

\maketitle
\section{}  % T1
from -16384 to 16383
\section{}  % T2
\begin{table}[H]
  \centering
  \begin{tabular}{ccc}
    \hline\hline
    X & Does the program halt? & Value stored in R0 \\
    \hline
    000000010 & Yes & 2 \\
    000000001 & Yes & 3 \\
    000000000 & Yes & 6 \\
    111111111 & No & - \\
    111111110 & No & - \\
    \hline\hline
  \end{tabular}
\end{table}
\section{}  % T3
The result of adding the values of R1 and R2 is not positive. 
\section{}  % T4
\begin{enumerate}
  \item No. Operate instructions are mainly about arithmetic operations and won't be affected by reduced registers.
  \item Yes. The reduced number of registers could lead to a more efficient use of the available registers. 
  \item No.
\end{enumerate}
\section{}  % T5
Data in memory cells:
\begin{table}[H]
  \centering
  \begin{tabular}{|c|c|c|}
    \hline
    1 & 0 & 1 \\
    \hline
    0 & 1 & 1 \\
    \hline
    0 & 1 & 0 \\
    \hline
    1 & 1 & 0 \\
    \hline
  \end{tabular}
\end{table}
\[D_{out}[2:0] = 1\ 1\ 0\]
\section{}  % T6

\section{}  % T7
\begin{enumerate}
  \item $2\times 10^8$
  \item $2.5 \times 10^7$
  \item $2 \times 10^8$
\end{enumerate}
\section{}  % T8
\begin{table}[H]
  \centering
  \begin{tabular}{cc}
    \hline\hline
    Address & Instruction \\
    \hline
    x3000 & 1001\ 111\ 001\ 111111 \\
    x3001 & 1001\ 001\ 101\ 111111 \\
    x3002 & 0101\ 101\ 111\ 000\ 010 \\
    x3003 & 0101\ 100\ 110\ 000\ 001 \\
    x3004 & 1001\ 001\ 101\ 111111 \\
    x3005 & 1001\ 010\ 100\ 111111 \\
    x3006 & 0101\ 000\ 001\ 000\ 010 \\
    x3007 & 1001\ 011\ 000\ 111111 \\
    \hline\hline
  \end{tabular}
\end{table}
\section{}  % T9
\begin{enumerate}
  \item Add the value of R1 and 00010 (sign-extended to 16 bits) then store the result in R2.
  \item Simply increment the PC under any condition. This instruction could be used for NOP.
  \item Execute the instruction of incremented PC plus 4 if the result is non-zero, otherwise execute the instruction of incremented PC.
  \item Bit-wise complement the value of R7 and store the result in R2.
  \item Input a character from the keyboard.
\end{enumerate}
\section{}  % T10
When we use the BR instruction, the offset is a 9-bit number, 
so the range of offset is from -256 to 255. 
If the instruction we want to jump is too far away, the offset will be out of range.
But the JMP instruction jumps to the address stored in the register.
Since the address is a 16-bit number, it can jump to any address.

\end{document}
\iffalse
\begin{figure}[h]
    \centering
    \includegraphics[scale=0.5]{name.png}
    \caption{name}
\end{figure}
\fi