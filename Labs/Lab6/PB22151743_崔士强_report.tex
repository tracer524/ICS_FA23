\documentclass[UTF8]{ctexart}
\usepackage[a4paper,left=3cm,right=3cm,top=2cm]{geometry}
\usepackage{amsmath}
\usepackage{enumitem}
\usepackage{float}
\usepackage{threeparttable}
\usepackage{caption}
\usepackage{multirow}
\usepackage{graphicx}
\usepackage{listings}
\usepackage{xcolor}
\usepackage{amssymb}
\usepackage[colorlinks,linkcolor=blue]{hyperref}
\renewcommand{\figurename}{Figure}
\definecolor{dkgreen}{rgb}{0,0.6,0}
\definecolor{gray}{rgb}{0.5,0.5,0.5}
\definecolor{mauve}{rgb}{0.58,0,0.82}
\lstdefinelanguage{LC3}{
  morekeywords={ADD, AND, BR, BRn, BRnz, BRnzp, BRnp, BRz, BRzp, BRp, JMP, JSR, LD, LDI, LDR, LEA, NOT, ST, STI, STR},
  sensitive=false,
  morecomment=[l]{;},
  morestring=[b]",
  morekeywords=[2]{R0, R1, R2, R3, R4, R5, R6, R7}, % Define register keywords
}
\lstset{frame=tb,
  language=LC3,
  aboveskip=3mm,
  belowskip=3mm,
  showstringspaces=false,
  columns=flexible,
  basicstyle={\small\ttfamily},
  numbers=left,%设置行号位置none不显示行号
  numberstyle=\tiny\courier, %设置行号大小
  numberstyle=\tiny\color{gray},
  keywordstyle=\color{blue},
  keywordstyle=[2]\color{purple},
  commentstyle=\color{dkgreen},
  stringstyle=\color{mauve},
  breaklines=true,
  breakatwhitespace=true,
  escapeinside=`,%逃逸字符(1左面的键),用于显示中文例如在代码中`中文...`
  tabsize=4,
  extendedchars=false %解决代码跨页时,章节标题,页眉等汉字不显示的问题
}

\setlength\lineskiplimit{5.25bp}
\setlength\lineskip{5.25bp}

\title{Lab06 Report}
\author{崔士强 PB22151743}
\date{December 31, 2023}

\begin{document}

\maketitle
\section{Purpose}
The purpose of the program is to compute the factorial of a non-negative number $N$, 

Anticipated outcomes: The program starts with repeatedly printing the student ID, waiting for the input.
For differnt kinds of input, the output varies. 
First the program tells whether the input is a decimal digit. If yes, the output is as follows:
\begin{enumerate}
  \item $0 \leq N \leq 7$ : $N!$ is displayed.
  \item $8 \leq N \leq 9$ : print: “$N!$ is too large for LC-3.”
\end{enumerate}

\section{Principles}
\subsection{Calling ISR}
In the OS code, bit 14 of the KBSR is set to 1. When input is detected, bit 15 of the KBSR is set to 1,
enabling the interrupt signal. Then the interrupt service routine is called.
\subsection{Getting literal value of input}
When we use \lstinline{GETC} to get the input character, the ASCII value is stored in \lstinline{R0}.
We also need its original value.

During checking the range of input, the instructions below are performed:
\begin{lstlisting}
  LD      R1, ASCII_0
  ADD     R1, R0, R1
  BRn     NON_DEC         ; Smaller than '0'
\end{lstlisting}
Here \lstinline{ASCII_0} is \lstinline{#-48}. After the \lstinline{ADD} instruction, 
the value in R1 is the literal value of the input. So we store it in another register for possible future use.
\subsection{Giving the results}
In this program, a trivial approach is used to give the final result: list all possibilities:
\begin{lstlisting}
            ADD     R3, R3, #-1
            BRn     RES0
            ADD     R3, R3, #-1
            BRn     RES1
            ; RES2 through RES7 are similar and not displayed here
RES0        LEA     R0, MSG_RES0
            PUTS
            BRnzp   RETURN
RES1        LEA     R0, MSG_RES1
            PUTS
            BRnzp   RETURN
\end{lstlisting}

\section{Procedure}
\subsection{Bugs encountered}
When testing the program, I saw "Access violation" in the console. After debugging, I found 
out that the bug was caused by inappropriate instruction to print strings. To print strings
using PUTS, the starting address of the string must be loaded into \lstinline{R0}. Example:
\begin{lstlisting}
  LEA     R0, MSG_EXCD
  PUTS
  MSG_EXCD    .STRINGZ    "! is too large for LC-3."
\end{lstlisting}
But I mistakenly used \lstinline{LD}.

Solution: use \lstinline{LEA} instead.

\subsection{Chanllenges}
A major challenge in the task is to print the result.

Reason for its difficulty is that the result may be more than 1 digit. For a number
like this, we can not directly convert it to ASCII value. Possible procedures to print the result
are as follows:
\begin{enumerate}
  \item Divide the result by $10$
  \item Obtain the remainder and substract this remainder from the result
  \item Save the remainder
  \item Repeat step 1 if the result is positive
  \item Print the remainders in reverse order
\end{enumerate}
Due to limited time, this approach, though more like “computing”, is not adopted.

\section{Results}
Screen recording: \href{https://rec.ustc.edu.cn/share/8b6ad340-a749-11ee-a899-95416037a7ff}{Lab06 Program Test} 

\end{document}
